\documentclass{beamer}
\usepackage[latin1]{inputenc}
\usetheme{Warsaw}
\usepackage{listings}
\usepackage{relsize}

\title{Port scans detector with prolog}
\author{Andrea Imparato - Lorenzo Tessari}\institute{Unversit\`a degli studi di Padova}

\begin{document}

\begin{frame}
\titlepage
\end{frame}



\begin{frame}
\frametitle{Cos'\`e?}
Software che riconosce la presenza 
di un port scanning su di un host in rete.\\

\textbf{Port scanning} = enumerazione porte attive/filtrate/chiuse\\[2cm]

2 tipologie di scanning:

\begin{itemize}

\item tcp scan

\item syn scan


\end{itemize}




\end{frame}



\begin{frame}
\frametitle{Motivazioni}
\begin{itemize}[<+->]
\item Tema sicurezza: conoscenza personale tool di sicurezza come \textbf{nmap} (port scanning) e \textbf{wireshark} (sniffing)
\item Snort IDS modulo port scanning non adeguato
\item Prolog!
\end{itemize}
\end{frame}


\begin{frame}
\frametitle{Use case}
 \includegraphics[width=10cm]{usecase1.png}
\end{frame}




\begin{frame}
\frametitle{Diagramma delle attivit\`a}
\begin{center}
 \includegraphics[height=8cm]{attivita.png}
\end{center}
\end{frame}



\begin{frame}
\frametitle{Progettazione}
Librerie:

\begin{itemize}
\item jpcap $\Rightarrow$ sniffer
\item tuprolog $\Rightarrow$ KB

\end{itemize}


\begin{center}
 \includegraphics[height=5cm]{classi.png}
\end{center}
\end{frame}



\begin{frame}
\frametitle{Protocollo Tcp/Ip}
\begin{center}
 \includegraphics[width=5cm,height=5cm]{tcp.png}
\end{center}
\end{frame}



\begin{frame}[fragile]
\frametitle{Inferenza - Riconoscimento connessione tcp}
\begin{lstlisting}
/* regola per connessione tcp */
connessione_tcp(SOURCE,DESTINATION,SP,DP):-
pacchetto(SP,DP,syn,SOURCE,DESTINATION,X,0)
,pacchetto(DP,SP,syn,DESTINATION,SOURCE,Y,Z
,pacchetto(SP,DP,SOURCE,DESTINATION,Z,W)
,Z is X+1,W is Y+1.
\end{lstlisting}


\end{frame}


\begin{frame}[fragile]
\frametitle{Inferenza - Riconoscimento connessione syn}
\begin{lstlisting}
/* regola per connessione connessione syn */
connessione_syn(SOURCE,DESTINATION,SP,DP):-
pacchetto(SP,DP,syn,SOURCE,DESTINATION,X,0)
,pacchetto(DP,SP,syn,DESTINATION,SOURCE,Y,Z)
,Z is X+1.

\end{lstlisting}
\end{frame}



\begin{frame}[fragile]
\frametitle{Inferenza - Riconoscimento porta chiusa}
\begin{lstlisting}
/* regola per riconoscere se la porta e' chiusa */
porta_chiusa(SOURCE,DESTINATION,SP,DP):-
pacchetto(SP,DP,syn,SOURCE,DESTINATION,X,0)
,pacchetto(DP,SP,rst,DESTINATION,SOURCE,0,Z)
,Z is X+1.

\end{lstlisting}
\end{frame}


\begin{frame}[fragile]
\frametitle{Inferenza scan 4 porte}
\begin{lstlisting}
tcp_scan(X,Y):- 
connessione_tcp(X,Y,A1,A2),A1\=A2,
connessione_tcp(X,Y,A3,A4),
A3\=A4,connessione_tcp(X,Y,A5,A6)
,A5\=A6,connessione_tcp(X,Y,A7,A8)
,A7\=A8,A2\=A4,A2\=A6,A2\=A8,A4\=A6
,A4\=A8,A6\=A8.

syn_scan(X,Y):- 
connessione_syn(X,Y,A1,A2)
,A1\=A2,connessione_syn(X,Y,A3,A4),A3\=A4
,connessione_syn(X,Y,A5,A6),A5\=A6,
connessione_syn(X,Y,A7,A8),A7\=A8,A2\=A4,A2\=A6,
A2\=A8,A4\=A6,A4\=A8,A6\=A8.
\end{lstlisting}
\end{frame}


\begin{frame}
\frametitle{Risultati e sviluppi futuri}

\begin{itemize}[<+->]
\item In presenza di poco traffico sulla rete prestazioni ottime. Zero risultati di falsi positivi/negativi
sia per tcp scan sia syn scan.
\item Syn scan difficile da trovare con molto traffico. Euristiche nmap? 
\item Tcp scan quasi banale anche con molto traffico.
\end{itemize}

\end{frame}




\begin{frame}
\begin{center}
{\huge \textbf{DEMO!}}
\end{center}
\end{frame}




\end{document}
